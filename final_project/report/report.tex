\documentclass[%
reprint,
%superscriptaddress,
%groupedaddress,
%unsortedaddress,
%runinaddress,
%frontmatterverbose,
%preprint,
%showpacs,preprintnumbers,
%nofootinbib,
%nobibnotes,
%bibnotes,
 amsmath,amssymb,
 aps,
%pra,
%prb,
%rmp,
%prstab,
%prstper,
%floatfix,
]{revtex4-2}

\usepackage{graphicx}% Include figure files
\usepackage{dcolumn}% Align table columns on decimal point
\usepackage{bm}% bold math
%\usepackage{hyperref}% add hypertext capabilities
%\usepackage[mathlines]{lineno}% Enable numbering of text and display math
%\linenumbers\relax % Commence numbering lines

%\usepackage[showframe, %Uncomment any one of the following lines to test
%scale=0.7, marginratio={1:1, 2:3}, ignoreall,% default settings
%text={7in,10in},centering,
%margin=1.5in,
%total={6.5in,8.75in}, top=1.2in, left=0.9in, includefoot,
%height=10in,a5paper,hmargin={3cm,0.8in},
%]{geometry}

%\usepackage[utf8]{inputenc}
%\usepackage[left=2.5 cm, right = 2.5 cm, top = 2.5 cm, bottom = %2cm, headheight=36pt]{geometry}
\usepackage{amsmath}
\usepackage{amssymb}
\usepackage{amsthm}
\usepackage{enumitem}
\usepackage{scrextend}
\usepackage{setspace}
\usepackage{fancyhdr}
\pagestyle{fancy}
\usepackage{siunitx}
%\usepackage[square,numbers]{natbib} 
%\bibliographystyle{abbrvnat}
%\usepackage{hyperref}
%\usepackage{cleveref}
%\usepackage{gensymb}


\begin{document}

\title{Worm algorithm for the Ising model}% Force line breaks with \\
%\thanks{A footnote to the article title}%

\author{Ajay S. Sakthivasan}
 \email{s6ajsakt@uni-bonn.de}
\author{Dongjin Suh}
 \email{s6sosuhh@uni-bonn.de}

\date{\today}% It is always \today, today,
             %  but any date may be explicitly specified

\begin{abstract}
  An article usually includes an abstract, a concise summary of the work
  covered at length in the main body of the article.
  \begin{description}
  \item[Usage]
    Secondary publications and information retrieval purposes.
  \item[Structure]
    You may use the \texttt{description} environment to structure your abstract;
    use the optional argument of the \verb+\item+ command to give the category of each item.
  \end{description}
\end{abstract}
\maketitle

\tableofcontents

\section{Introduction} 
Monte Carlo methods are powerful numerical techniques for simulating complex systems. In the context of the Ising model, Monte Carlo methods are used to sample from the Boltzmann distribution. The Metropolis algorithm is a widely used Monte Carlo method for the Ising model. The local spin-flip Metropolis-algorithm, which is one of this methods, is a widely used technique for generating statistically independent configurations in statistical physics simulations. \\
However, when approaching the pseudocritical regime, this algorithm is known to suffer from an explosion of computational cost, making it impractical to produce large numbers of uncorrelated configurations. This phenomenon is called critical slowing down, and it is quantified by the dynamical exponent z of the algorithm. The value of z describes the volume scaling of the autocorrelation time, which measures how long it takes for a configuration to become decorrelated from its initial state. \\
To address this problem, Prokof’ev and Svistunov proposed an alternative update algorithm within the Metropolis scheme, which has since become known as the “Worm”-algorithm. The Worm algorithm preserves the local nature of the update step, which is important for many physical systems, but achieves a very small dynamical exponent. As a result, it avoids critical slowing down and allows for efficient generation of uncorrelated configurations. \\
In this paper, we will discuss the Worm algorithm and how it is used to simulate the Ising model. We will start by discussing the basic principles of the Ising model and the Monte Carlo simulation. The next step is the implementing of worm algorithm for the Ising model in 2 dimensions and comparing the scaling of autocorrelation times to the single-spin Metropolis update. In addition we also want to determine the critical exponents. Then we will extend the worm algorithm by considering the Ising model in 3 dimensions.    


\section{Theoretical basis}
\subsection{Ising model} 
The Ising model is a mathematical model that has been extensively used in statistical mechanics to understand the behavior of ferromagnetism. The model consists of a lattice of magnetic moments that can either be up or down. These magnetic moments interact with their neighboring magnetic moments and tend to align with their neighboring spins. The Ising model is used to understand the behavior of systems that exhibit phase transitions, like ferromagnetic materials, where a change in temperature or an external magnetic field can lead to a sudden change in the behavior of the system. \\ 
First we want to introduce the Ising model in 2 dimensions. It is used to describe the behavior of magnetic systems consisting of interacting spins on a two-dimensional lattice. This system of spins is immersed in a heat bath of constant temperature T and in  an external magnetic field h. In this model, each spin can be in one of two states (up or down) and interacts with its nearest neighbors. The strength of the interaction is determined by a coupling constant, J, and the system is described by a Hamiltonian. The Hamiltonian of the 2D Ising model is given by: 
\begin{align}
    H = J\sum_{\langle i,j \rangle} s_{i}s_{j} - h\sum_{i} s_{i}
\end{align}
The variables $s_i$ and $s_j$ are the spin variables, which can take on the values of +1 or -1.

\subsubsection{Partition function}
The partition function is a sum over all possible spin configurations s
\begin{align}
    Z = \sum_{s} \mathrm{e}^{-\beta H} = \sum_{s} \mathrm{e}^{-[-\beta J \sum_{\langle i,j \rangle}s_i s_j - \beta h \sum_{i}s_i]}
\end{align}
where $\beta$ is the inverse thermodynamic temperature, $\beta = (k_{b}T)^{-1}$


\subsubsection{The critical exponents}
The critical exponents of the Ising model describe how the properties of the system change near the critical point, where the system undergoes a phase transition.
There are several critical exponents associated with the Ising model, including:

One of the easiest properties to determine, especially numerically, is the magnetisation per spin, M. The critical exponent for the magnetization, denoted as $\beta$. This exponent describes how the magnetization of the system changes near the critical point. This is simply given by 
\begin{align}
    M = \frac{1}{N}\sum_{i}^{N} \sigma_{i}
\end{align}

with N being the amount of sites. For a square lattice one simply replaces N by $L^{2}$, where L is the lattice length.

The critical exponent for the correlation length, denoted as $\nu$. This exponent describes how the correlation length of the system diverges near the critical point. Specifically, the correlation length scales as $\xi \sim |t|^{-\nu}$, where $t$ is the reduced temperature (the difference between the actual temperature and the critical temperature).

The critical exponent for the susceptibility, denoted as $\gamma$. This exponent describes how the susceptibility of the system diverges near the critical point. 
\begin{align}
    \chi = \frac{M}{H} 
\end{align}
Specifically, the susceptibility scales as $\chi \sim |t|^{-\gamma}$.

The critical exponent for the specific heat, denoted as $\alpha$. This exponent describes how the specific heat of the system diverges near the critical point. 
\begin{align}
    C = (k_B \beta)^2 \cdot (\langle E^2 \rangle)-(\langle E \rangle)^2) 
\end{align}
Specifically, the specific heat scales as $C \sim |t|^{-\alpha}$.


\section{Methods}
\subsection{Monte Carlo methods - Metropolis algorithm}
The Metropolis algorithm is a Monte Carlo simulation method used to generate samples from a probability distribution, in this case, the Boltzmann distribution of spin configurations. The Ising model describes a collection of magnetic moments (spins) that can be in an up or down state. The energy of the system depends on the interaction between neighboring spins, as well as an external magnetic field. In the Metropolis algorithm, the system is updated iteratively by flipping a random spin and accepting or rejecting the new state based on the change in energy. \\
To implement the Metropolis algorithm to simulate the 2D Ising model, we went through the following steps: \\
First of all a random spin is selected from the spin configuration of the square lattice since we consider the 2D Ising model. 
Next, the energy change associated with flipping this spin is calculated using the toroidal geometry of the Ising model.  
The change in energy by flipping the spin is used to determine whether to accept or reject the spin flip, based on the Metropolis acceptance probability. The new state is accepted with probability $\exp{-beta*dE}$ if the energy decreases or with probability 1 if the energy increases. 
Thetotal spin and energy of the spin configuration are saved for each iteration. 
We include some burnin iterations to get the thermal equilibrium before we start actual iterations and save the calculated properties.   


\subsection{worm algorithm}
However, the Metropolis algorithm can suffer from critical slowing down. So we want introduce a new algorithm, which is called the Worm algorithm. The worm algorithm is a powerful Monte Carlo method and operates on graph configurations instead of individual spins. This way, the algorithm can stay local and can avoid, to some extent, the critical slowdown that happens near transition points.
The worm algorithm was first introduced by Prokof'ev and Svistunov in 2001. The worm algorithm is based on the concept of a worldline, which is a path in time and space that describes the history of a spin flip. The worldline can be used to construct a worm, which is a worldline with a head and a tail. The worm algorithm is based on the idea of creating and destroying worms, and it can be used to efficiently sample from the Ising model.
The algorithm still uses the Metropolis acceptance rates and therefore it fulfills detailed balance. \\
The implementation of the worm algorithm starts initializing the size of the Ising model and arrays to store the total number of spins and total energy at each iteration. It then runs a burn-in phase to allow the system to reach equilibrium. \\
Next, the code enters a loop that runs for the specified number of iterations. In each iteration, a new worm is created by randomly selecting a starting position on the lattice. The worm then moves through the lattice by randomly selecting a direction to move in and checking whether the move is valid. 
.....
IF the move is valid, the energy cost of flipping the spin at the new position is calculated, and the spin is either flipped or not based on a Metropolis criterion. The positon of the worm is then updated. 
If one of the three cases appears, the one iteration will end: 
-The head of the worm meets the tail
-The head moves out of the space of spin configuration
-The Metropolis step rejects the spin flip
After each iteration the total spin and the energy of the current spin configuration are determined.  



\subsection{Determining critical exponents}



\subsection{Error analysis - Bootstrapping}
The bootstrap method is a statistical technique used to estimate the sampling distribution of a statistic. It is particularly useful when the population distribution is unknown or complex, and it can be used to estimate the standard error of a statistic or to construct confidence intervals. \\
The bootstrap method involves resampling from the original sample to create multiple new samples, each of which has the same size as the original sample. For each new sample, a statistic of interest (e.g. mean, median, standard deviation) is calculated. By repeating this resampling process many times, we obtain a distribution of the statistic, known as the bootstrap distribution. \\
This bootstrap distribution can be used to estimate the standard error of the statistic, which reflects the variability of the statistic across different samples. Specifically, the standard deviation of the bootstrap distribution is an estimate of the standard error of the statistic.

We implemented the bootstraop method like following: \\
This function performs the bootstrap method to estimate the mean and standard deviation of a given dataset data. It takes two arguments: the dataset data and the number of bootstrap samples.
First, the function initializes an array of zeros with length of the bootstrap samples. This array will be used to store the means of each bootstrap sample. \\
Next, a for loop is used to generate bootstrap samples. For each bootstrap sample, a new configuration of the markov chain is chosen randomly using the np.random.randint function. Then we have an array that contains the indices of the data points selected for the bootstrap sample. \\
The function then creates a new data configuration by selecting the data points with the indices.
Finally, the mean of the bootstrap sample can be calculated and stored.
After generating all the bootstrap samples, the function returns the mean and standard deviation of all bootstrap samples means. So the at the end returned mean is an estimate of the population mean, while the returned standard deviation is an estimate of the standard error of the mean.

\section{Results}
\subsection{magnetisation}
\subsection{Energy}
\subsection{integrated Autocorrelation time}
\subsection{correlation length}

\section{Discussion}

\section{Summary}


\section{\label{sec:level1}First-level heading}

This sample document was adapted from the template for papers
in APS journals.
It demonstrates proper use of REV\TeX~4.1 (and
\LaTeXe) in mansucripts prepared for submission to APS
journals. Further information can be found in the REV\TeX~4.1
documentation included in the distribution or available at
\url{http://authors.aps.org/revtex4/}.

When commands are referred to in this example file, they are always
shown with their required arguments, using normal \TeX{} format. In
this format, \verb+#1+, \verb+#2+, etc. stand for required
author-supplied arguments to commands. For example, in
\verb+\section{#1}+ the \verb+#1+ stands for the title text of the
author's section heading, and in \verb+\title{#1}+ the \verb+#1+
stands for the title text of the paper.

Line breaks in section headings at all levels can be introduced using
\textbackslash\textbackslash. A blank input line tells \TeX\ that the
paragraph has ended. Note that top-level section headings are
automatically uppercased. If a specific letter or word should appear in
lowercase instead, you must escape it using \verb+\lowercase{#1}+ as
in the word ``via'' above.

\subsection{\label{sec:level2}Second-level heading: Formatting}

This file may be formatted in either the \texttt{preprint} or
\texttt{reprint} style. \texttt{reprint} format mimics final journal output.
Either format may be used for submission purposes. \texttt{letter} sized paper should
be used when submitting to APS journals.

\subsubsection{Wide text (A level-3 head)}
The \texttt{widetext} environment will make the text the width of the
full page, as on page~\pageref{eq:wideeq}. (Note the use the
\verb+\pageref{#1}+ command to refer to the page number.)
\paragraph{Note (Fourth-level head is run in)}
The width-changing commands only take effect in two-column formatting.
There is no effect if text is in a single column.

\subsection{\label{sec:citeref}Citations and References}
A citation in text uses the command \verb+\cite{#1}+ or
\verb+\onlinecite{#1}+ and refers to an entry in the bibliography.
An entry in the bibliography is a reference to another document.

\subsubsection{Citations}
Because REV\TeX\ uses the \verb+natbib+ package of Patrick Daly,
the entire repertoire of commands in that package are available for your document;
see the \verb+natbib+ documentation for further details. Please note that
REV\TeX\ requires version 8.31a or later of \verb+natbib+.

\paragraph{Syntax}
The argument of \verb+\cite+ may be a single \emph{key},
or may consist of a comma-separated list of keys.
The citation \emph{key} may contain
letters, numbers, the dash (-) character, or the period (.) character.
New with natbib 8.3 is an extension to the syntax that allows for
a star (*) form and two optional arguments on the citation key itself.
The syntax of the \verb+\cite+ command is thus (informally stated)
\begin{quotation}\flushleft\leftskip1em
  \verb+\cite+ \verb+{+ \emph{key} \verb+}+, or\\
  \verb+\cite+ \verb+{+ \emph{optarg+key} \verb+}+, or\\
  \verb+\cite+ \verb+{+ \emph{optarg+key} \verb+,+ \emph{optarg+key}\ldots \verb+}+,
\end{quotation}\noindent
where \emph{optarg+key} signifies
\begin{quotation}\flushleft\leftskip1em
  \emph{key}, or\\
  \texttt{*}\emph{key}, or\\
  \texttt{[}\emph{pre}\texttt{]}\emph{key}, or\\
  \texttt{[}\emph{pre}\texttt{]}\texttt{[}\emph{post}\texttt{]}\emph{key}, or even\\
  \texttt{*}\texttt{[}\emph{pre}\texttt{]}\texttt{[}\emph{post}\texttt{]}\emph{key}.
\end{quotation}\noindent
where \emph{pre} and \emph{post} is whatever text you wish to place
at the beginning and end, respectively, of the bibliographic reference
(see Ref.~[\onlinecite{witten2001}] and the two under Ref.~[\onlinecite{feyn54}]).
(Keep in mind that no automatic space or punctuation is applied.)
It is highly recommended that you put the entire \emph{pre} or \emph{post} portion
within its own set of braces, for example:
\verb+\cite+ \verb+{+ \texttt{[} \verb+{+\emph{text}\verb+}+\texttt{]}\emph{key}\verb+}+.
The extra set of braces will keep \LaTeX\ out of trouble if your \emph{text} contains the comma (,) character.

The star (*) modifier to the \emph{key} signifies that the reference is to be
merged with the previous reference into a single bibliographic entry,
a common idiom in APS and AIP articles (see below, Ref.~[\onlinecite{epr}]).
When references are merged in this way, they are separated by a semicolon instead of
the period (full stop) that would otherwise appear.

\paragraph{Eliding repeated information}
When a reference is merged, some of its fields may be elided: for example,
when the author matches that of the previous reference, it is omitted.
If both author and journal match, both are omitted.
If the journal matches, but the author does not, the journal is replaced by \emph{ibid.},
as exemplified by Ref.~[\onlinecite{epr}].
These rules embody common editorial practice in APS and AIP journals and will only
be in effect if the markup features of the APS and AIP Bib\TeX\ styles is employed.

\paragraph{The options of the cite command itself}
Please note that optional arguments to the \emph{key} change the reference in the bibliography,
not the citation in the body of the document.
For the latter, use the optional arguments of the \verb+\cite+ command itself:
\verb+\cite+ \texttt{*}\allowbreak
\texttt{[}\emph{pre-cite}\texttt{]}\allowbreak
\texttt{[}\emph{post-cite}\texttt{]}\allowbreak
\verb+{+\emph{key-list}\verb+}+.

\end{document}
%
% ****** End of file templateForReport.tex ******